% _____________________________________________________________________________
% *****************************************************************************
% Autor: José Antonio Quiñonero Gris
% Fecha de creación: Friday 20:00:58 28-10-2022
% *****************************************************************************
% -----------------------------------------------------------------------------
\documentclass[12pt, a4paper]{article}

\usepackage[utf8]{inputenc}
\usepackage[T1]{fontenc}
\usepackage{textcomp}
\usepackage[spanish, es-tabla]{babel}
\usepackage{amsmath, amssymb}

\input{/home/jose/Documents/latex/preamble/paquetes1} %mis paquetes
% \usepackage[backend=biber,style=chem-acs,terseinits=true,sorting=none,isbn=false,doi=false]{biblatex}
% \addbibresource{/home/jose/Documents/latex/preamble/references} % extension must be written
%%%%
\usepackage[backend=bibtex,style=chem-acs,terseinits=true,sorting=none,isbn=false,doi=false]{biblatex}
% \usepackage[backend=bibtex,style=chem-acs,terseinits=true,sorting=none,isbn=false,doi=false]{biblatex}
\bibliography{/home/jose/Documents/latex/preamble/references}
%%%
\ExecuteBibliographyOptions{%
  citetracker=true,% Citation tracker enabled in order not to repeat citations, and have two lists.
  sorting=none,% Don't sort, just print in the order of citation
  alldates=long,% Long dates, so we can tweak them at will afterwards
  dateabbrev=false,% Remove abbreviations in dates, for same reason as ``alldates=long''
  articletitle=true,% To have article titles in full bibliography
  maxcitenames=999% Number of names before replacing with et al. Here, everyone.
  }

% No brackets around the number of each bibliography entry
\DeclareFieldFormat{labelnumberwidth}{#1\addperiod}

% Suppress article title, doi, url, etc. in citations
\AtEveryCitekey{%
  \ifentrytype{article}
    {\clearfield{title}}
    {}%
  \clearfield{doi}%
  \clearfield{url}%
  \clearlist{publisher}%
  \clearlist{location}%
  \clearfield{note}%
}

% Print year instead of date, when available; make use of urldate
\DeclareFieldFormat{urldate}{\bibstring{urlseen}\space#1}
\renewbibmacro*{date}{% Based on date bib macro from chem-acs.bbx
  \iffieldundef{year}
    {\ifentrytype{online}
       {\printtext[urldate]{\printurldate}}
       {\printtext[date]{\printdate}}}
    {\printfield[date]{year}}}

% Remove period from titles
\DeclareFieldFormat*{title}{#1}
% Make year bold for @book types
\DeclareFieldFormat[book]{date}{\textbf{#1}} % doctorate added this line
\DeclareFieldFormat[book]{title}{\textit{#1}} % doctorate added this line
\DeclareFieldFormat[book]{publisher}{#1,} % doctorate added this line
% Embed doi and url in titles, when available
\renewbibmacro*{title}{% Based on title bib macro from biblatex.def
  \ifboolexpr{ test {\iffieldundef{title}}
               and test {\iffieldundef{subtitle}} }
    {}
    {\ifboolexpr{ test {\ifhyperref}
                  and not test {\iffieldundef{doi}} }
       {\href{http://dx.doi.org/\thefield{doi}}
          {\printtext[title]{%
             \printfield[titlecase]{title}%
             \setunit{\subtitlepunct}%
             \printfield[titlecase]{subtitle}}}}
       {\ifboolexpr{ test {\ifhyperref}
                     and not test {\iffieldundef{url}} }
         {\href{\thefield{url}}
            {\printtext[title]{%
               \printfield[titlecase]{title}%
               \setunit{\subtitlepunct}%
               \printfield[titlecase]{subtitle}}}}
         {\printtext[title]{%
            \printfield[titlecase]{title}%
            \setunit{\subtitlepunct}%
            \printfield[titlecase]{subtitle}}}}%
     \newunit}%
  \printfield{titleaddon}%
  \clearfield{doi}%
  \clearfield{url}%
  \clearfield{pagetotal}%
  \clearlist{language}% doctorate added this
  \clearfield{note}% doctorate added this
  \ifentrytype{article}% Delimit article and journal titles with a period
    {\adddot}
    {}}
 %configuracion para l bibliografia
\input{/home/jose/Documents/latex/preamble/paquetesTikz} %mis paquetes
\input{/home/jose/Documents/latex/preamble/tikzstyle} %Estilo para las gráficas

\decimalpoint

% \graphicspath{{../figuras/}}

\author{José Antonio Quiñonero Gris}
\title{
    \textbf{Notes on the derivatives of Lotka-Volterra equations}
      }

% \date{\today}

\date{
November 26, 2022
}

\begin{document}
\maketitle

% #############################################################################
% --- INICIO DEL DOCUMENTO ---
% #############################################################################

Considering the Lotka-Volterra equations:
\begin{align}
    \dfrac{\dd{x}}{\dd{t}} &= - \alpha x + \alpha' x^2 + \beta x y \\
    \dfrac{\dd{y}}{\dd{t}} &= \kappa y - \kappa' y^2 - \lambda x y
\end{align}

we can write them as
\begin{equation}\label{ec:main}
    \dfrac{\dd{y}}{\dd{t}} = y' =
    \underbrace{a y}_{y_1'}
    + \underbrace{b y^2}_{y_2'}
    + \underbrace{c x y}_{y_3'}
\end{equation}
following the logistic model. If $\alpha' = \kappa' = 0$, then we get the simple model.

For the Taylor method, we need to compute the successive derivatives. In order to find an expression that we can compute, let's first write the first derivatives by hand
% To do so, we have to follow the chain rule
% \begin{equation}
%     \dfrac{ \dd{y'} }{\dd{t}} = \dfrac{\dd{f(x,y)}}{\dd{t}} = \frac{\partial f}{\partial x} \frac{\partial x}{\partial t} + \frac{\partial f}{\partial y} \frac{\partial y}{\partial t}
% \end{equation}
% so
% \begin{equation}
%     \dfrac{ \dd{y'} }{\dd{t}} = y^{(2)} =
%     \underbrace{\frac{\partial y_1'}{\partial x} \frac{\partial x}{\partial t} + \frac{\partial y_1'}{\partial y} \frac{\partial y}{\partial t}}_{y_1^{(2)}} +
%     \underbrace{\frac{\partial y_2'}{\partial x} \frac{\partial x}{\partial t} + \frac{\partial y_2'}{\partial y} \frac{\partial y}{\partial t}}_{y_2^{(2)}} +
%     \underbrace{\frac{\partial y_3'}{\partial x} \frac{\partial x}{\partial t} + \frac{\partial y_3'}{\partial y} \frac{\partial y}{\partial t}}_{y_3^{(2)}}
% \end{equation}
as

\begin{equation}
    y^{(n+1)} = \dfrac{ \dd{y^{(n)}} }{\dd{t}} =
    \dfrac{ \dd{y_1^{(n)}} }{\dd{t}} +
    \dfrac{ \dd{y_2^{(n)}} }{\dd{t}} +
    \dfrac{ \dd{y_3^{(n)}} }{\dd{t}} =
    y_1^{(n+1)} + y_2^{(n+1)} + y_3^{(n+1)}
\end{equation}
where $y^{(n)}$ is the $n$-th derivative of $y$ with respect to $t$.

I am going to calculate the $n$ derivatives of each $y^{(n)}_i$ term separately. The first one, $y_1^{(n)}$, is:
\begin{align*}
    y_1^{(1)} &= ay \\
    y_1^{(2)} &= ay^{(1)} \\
    y_1^{(3)} &= ay^{(2)} \\
    y_1^{(4)} &= ay^{(3)} \\
    y_1^{(5)} &= ay^{(4)} \\
              &\shortvdotswithin{=} \notag
\end{align*} \\[-10mm]
so we can easily see that
\begin{equation}\label{ec:primer_termino}
    y_1^{(n+1)} = a y^{(n)}
\end{equation}

Let's skip the second term for now and focus on the third one, $y_3^{(n)}$:

\begin{align*}
    y_3^{(1)} &= c x y \\
    y_3^{(2)} &= c \left[ x y^{(1)} + x^{(1)} y \right] \\
    y_3^{(3)} &= c \left[ x y^{(2)} + 2 x^{(1)} y^{(1)} + x^{(2)} y \right] \\
    y_3^{(4)} &= c \left[ x y^{(3)} + 3 x^{(1)} y^{(2)} + 3 x^{(2)} y^{(1)} + x^{(3)} y \right] \\
    y_3^{(5)} &= c \left[ x y^{(4)} + 4 x^{(1)} y^{(3)} + 6 x^{(2)} y^{(2)} + 4 x^{(3)} y^{(1)} + x^{(4)} y \right] \\
              &\shortvdotswithin{=} \notag
\end{align*} \\[-15mm]

I didn't found it so trivial to find a general expresion for $y_3^{(n)}$. The approach I followed was, first, to write the terms inside brackets in columns, ordering them by increasing $(n)$ of $y^{(n)}$, as
\begin{alignat*}{3}
    y_3^{(1)} &= c x y \\
    y_3^{(2)} &= c \left[ x^{(1)} y + x y^{(1)}  \right] \\
    y_3^{(3)} &= c \left[ x^{(2)} y + 2 x^{(1)} y^{(1)} + x y^{(2)}  \right] \\
    y_3^{(4)} &= c \left[ x^{(3)} y + 3 x^{(2)} y^{(1)} + 3 x^{(1)} y^{(2)} + x y^{(3)} \right] \\
    y_3^{(5)} &= c \left[ x^{(4)} y + 4 x^{(3)} y^{(1)} + 6 x^{(2)} y^{(2)} + 4 x^{(1)} y^{(3)} + x y^{(4)} \right] \\
              &\shortvdotswithin{=} \notag
\end{alignat*} \\[-10mm]
or, including the zero terms to make it more visual
\begin{alignat*}{5}
    y_3^{(1)} &= c \left[ 1 x y       \right. &&+ 0 x y^{(1)}       &&+ 0 x y^{(2)}       &&+ 0 x y^{(3)}       &&+ \left. 0 x y^{(4)} \right] \\
    y_3^{(2)} &= c \left[ 1 x^{(1)} y \right. &&+ 1 x y^{(1)}       &&+ 0 x y^{(2)}       &&+ 0 x y^{(3)}       &&+ \left. 0 x y^{(4)} \right] \\
    y_3^{(3)} &= c \left[ 1 x^{(2)} y \right. &&+ 2 x^{(1)} y^{(1)} &&+ 1 x y^{(2)}       &&+ 0 x y^{(3)}       &&+ \left. 0 x y^{(4)} \right] \\
    y_3^{(4)} &= c \left[ 1 x^{(3)} y \right. &&+ 3 x^{(2)} y^{(1)} &&+ 3 x^{(1)} y^{(2)} &&+ 1 x y^{(3)}       &&+ \left. 0 x y^{(4)} \right] \\
    y_3^{(5)} &= c \left[ 1 x^{(4)} y \right. &&+ 4 x^{(3)} y^{(1)} &&+ 6 x^{(2)} y^{(2)} &&+ 4 x^{(1)} y^{(3)} &&+ \left. 1 x y^{(4)} \right] \\
              &\shortvdotswithin{=} \notag
\end{alignat*} \\[-15mm]

I can write the coefficients of each column in a table, writing each $i$-th row for the $i$-th derivative, $y_3^{(i)}$, which depends on the $j$-th derivative, $y^{(j)}$, column $j$-th, by the coefficient $i,j$.
\begin{table}[h!]
    \ra{1.2} % Spacing btween lines of table
    % \caption{}
    \label{tab:coeficientes}
    \centering
    \begin{tabular}{@{}c|ccccc@{}}
                  & $y$ & $y^{(1)}$ & $y^{(2)}$ & $y^{(3)}$ & $y^{(4)}$ \\
        \midrule
        % $y$       & 1   & 0         & 0         & 0         & 0         \\
        $y^{(1)}$ & 1   & 0         & 0         & 0         & 0         \\
        $y^{(2)}$ & 1   & 1         & 0         & 0         & 0         \\
        $y^{(3)}$ & 1   & 2         & 1         & 0         & 0         \\
        $y^{(4)}$ & 1   & 3         & 3         & 1         & 0         \\
        $y^{(5)}$ & 1   & 4         & 6         & 4         & 1         \\
                  & \multicolumn{5}{@{}l@{}}{\hspace{5pt}%
                      \raisebox{.5\normalbaselineskip}{%
                      \rlap{$\underbrace{\hphantom{\mbox{%
                        $y$%
                \hspace*{\dimexpr4\arraycolsep+\arrayrulewidth}%
                        $y^{(1)}$%
                \hspace*{\dimexpr4\arraycolsep+\arrayrulewidth}%
                        $y^{(2)}$%
                \hspace*{\dimexpr4\arraycolsep+\arrayrulewidth}%
                        $y^{(3)}$%
                                                        }}}_{\mat{C}}$}}%
                    }
    \end{tabular}
\end{table}

Creating a coefficient matrix $\mat{C}$, which can be computed as
\begin{equation}
    \mat{C}(i+1, j+1) = \mat{C}(i,j) + \mat{C}(i,j+1), \qquad 1 \le i,j \le N-1
\end{equation}
% for the first $N$ derivatives
% (Note: I'm considering the first element of the matrix is $\mat{C}_{0,0} \equiv \mat{C}(0,0)$, indexing as coded in Fortran).
with the conditions
\begin{alignat}{2}
    \text{First column} &\to \mat{C}_{i,1} &&= 1 \quad 1 \le i \le N \\
    \text{First row}    &\to \mat{C}_{1,j} &&= 0 \quad 2 \le j \le N
\end{alignat}

I write the third term as
\begin{equation}\label{ec:tercer_termino}
    y_3^{(n+1)} = c \sum_{k=0}^{n} \mat{C}_{n,k}\, x^{(n-k)} y^{(k)}
\end{equation}

(Note: the indexing of $\mat{C}$ in the previous equation starts in 0,0 to simplify formulation)

An improvement would be to write $\mat{C}_{i,j}$ in terms of $n$, but I didn't find a way in a first look, so I'm leaving this at it is.

Now, I go back to the second term, $y_2$, the last one to compute. If instead of writing it as $y_2 = by^2$ we write it as $y_2 = byy$, it has the same form of $y_3$ but substituting $x$ for $y$. Then, just to simplify the code, I won't apply the commutative law for multiplication and calculate the derivatives as in the previous example
\begin{alignat*}{3}
    y_2^{(1)} &= b y y \\
    y_2^{(2)} &= b \left[ y^{(1)} y +   y y^{(1)}  \right] \\
    y_2^{(3)} &= b \left[ y^{(2)} y + 2 y^{(1)} y^{(1)} +   y y^{(2)}  \right] \\
    y_2^{(4)} &= b \left[ y^{(3)} y + 3 y^{(2)} y^{(1)} + 3 y^{(1)} y^{(2)} +   y y^{(3)} \right] \\
    y_2^{(5)} &= b \left[ y^{(4)} y + 4 y^{(3)} y^{(1)} + 6 y^{(2)} y^{(2)} + 4 y^{(1)} y^{(3)} + y y^{(4)} \right] \\
              &\shortvdotswithin{=} \notag
\end{alignat*} \\[-15mm]

Then, we can employ the same formulation as before and write $y_2^{(n)}$ as
\begin{equation}\label{ec:segundo_termino}
    y_2^{(n+1)} = b \sum_{k=0}^{n} \mat{C}_{n,k}\, y^{(n-k)} y^{(k)}
\end{equation}

Finally, putting together eqs. \eqref{ec:primer_termino}, \eqref{ec:segundo_termino} and \eqref{ec:tercer_termino}, I compute the derivatives as
\begin{align*}
    y^{(n+1)} &= y_1^{(n+1)} + y_2^{(n+1)} + y_3^{(n+1)} = \\
              &= a y^{(n)} + b \sum_{k=0}^{n} \mat{C}_{n,k}\, y^{(n-k)} y^{(k)} + c \sum_{k=0}^{n} \mat{C}_{n,k}\, x^{(n-k)} y^{(k)}
\end{align*}

\begin{equation}
    \boxed{
        \therefore\ y^{(n+1)} = a y^{(n)} + b \sum_{k=0}^{n} \mat{C}_{n,k}\, y^{(n-k)} y^{(k)} + c \sum_{k=0}^{n} \mat{C}_{n,k}\, x^{(n-k)} y^{(k)}
    }
\end{equation}
% #############################################################################
\end{document}
